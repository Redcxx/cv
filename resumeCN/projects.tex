%-------------------------------------------------------------------------------
%	SECTION TITLE
%-------------------------------------------------------------------------------
\cvsection{项目}


%-------------------------------------------------------------------------------
%	CONTENT
%-------------------------------------------------------------------------------
\begin{cventries}

%---------------------------------------------------------
  \cventry
    {个人开发者 - HTML/CSS,Javascript,和 GLSL} % Role
    {个人网站} % Event
    {伦敦, 英国} % Location
    {六月 2020 - 现在} % Date(s)
    {
      \begin{cvitems} % Description(s)
        \item {从头搭建了两个个人网站,运用各类技术和第三方库如 \href{https://reactjs.org/}{reactJs},\href{https://nextjs.org/}{nextJs},\href{https://threejs.org/}{threeJs},\href{https://sass-lang.com/}{sass},\href{https://postcss.org/}{postcss} 和 \href{https://pugjs.org/api/getting-started.html}{pugJs}。}
        \item 前往 \url{https://luoweilue.com} (简单一点) 或者 \url{https://weilueluo.com} (好玩一点)查看。
      \end{cvitems}
    }
    
%---------------------------------------------------------
  \cventry
    {个人开发者 - Python} % Role
    {基于Pytorch的机器学习} % Event
    {伦敦, 英国} % Location
    {十一月 2020 - 现在} % Date(s)
    {
      \begin{cvitems} % Description(s)
      \item {\textbf{流程优化} 为一家公司实现相关领域内各类主流的机器学习模型,以此辅助动画师作画(研究生毕业项目,进行中不允许透露太多)。}
      \item {\textbf{时间序列预测} 运用各类模型生成单/多声道的钢琴音乐包括LSTM,残差连接和注意力机制。可在 \href{https://github.com/Redcxx/Third-Year-Project-Report}{\textit{https://github.com/Redcxx/Third-Year-Project-Report}}查看报告,生成的音乐样本可在\href{https://soundcloud.com/kallzvx/sets}{\textit{https://soundcloud.com/kallzvx/sets}}查看。}
      \item {\textbf{图像分类} 基于Resnet50的迁移学习,用1000+图片分类10个标签,准确率达90\%。}
      \end{cvitems}
    }
    
  %---------------------------------------------------------
  \cventry
    {主要开发者 - C\#} % Role
    {第一人称虚拟现实射击游戏 - Bouncing Ray} % Event
    {伦敦, 英国} % Location
    {三月 2022 - 四月 2022} % Date(s)
    {
      \begin{cvitems} % Description(s)
        \item {带领4人团队用Unity基于Ubiq开发框架(\href{https://github.com/UCL-VR/ubiq}{\textit{https://github.com/UCL-VR/ubiq}})编写了一个第一人称虚拟现实射击游戏。项目可在 \href{https://github.com/HengyiWang/COMP0113-BouncingRay}{\textit{https://github.com/HengyiWang/COMP0113-BouncingRay}}查看。}
      \end{cvitems}
    }
  
  \cventry
    {个人开发者 - Python} % Role
    {本地DNS服务器} % Event
    {伦敦, 英国} % Location
    {二月 2022 - 三月 2022} % Date(s)
    {
      \begin{cvitems} % Description(s)
        \item {根据\href{https://datatracker.ietf.org/doc/html/rfc1034}{RFC 1034} 和 \href{https://datatracker.ietf.org/doc/html/rfc1035}{RFC 1035}标准,编写了一个本地DNS服务器,可以用于解析任意IPv4地址。项目可在 \href{https://github.com/Redcxx/local-nameserver}{\textit{https://github.com/Redcxx/local-nameserver}}查看。}
      \end{cvitems}
    }
    
  \cventry
    {Sole Developer - Python} % Role
    {Poisson Image Editing} % Event
    {London, UK} % Location
    {Dec. 2021 - Jan. 2022} % Date(s)
    {
      \begin{cvitems} % Description(s)
        \item {Implemented the \href{https://www.cs.jhu.edu/~misha/Fall07/Papers/Perez03.pdf}{Poisson Image Editing paper}. Including techniques such as naive filling; import \& mixing gradient; texture flattening, colour isolation \& illumination.  Project can be found at \href{https://github.com/Redcxx/poisson-image-editing}{\textit{https://github.com/Redcxx/poisson-image-editing}}.}
      \end{cvitems}
    }

%---------------------------------------------------------
  \cventry
    {Sole Developer - Python} % Role
    {Image Downloader} % Event
    {Manchester, UK} % Location
    {Jun. 2019 - Aug. 2019} % Date(s)
    {
      \begin{cvitems} % Description(s)
        \item {Designed and implemented an API supports automatic login, multi‑threading downloading, various format parsing and filter images, along with a graphical user interface. See \href{https://github.com/Redcxx/Pikax}{\textit{https://github.com/Redcxx/Pikax}}.}
      \end{cvitems}
    }
    
  \cventry
    {Sole Developer} % Role
    {Other} % Event
    {} % Location
    {} % Date(s)
    {
      \begin{cvitems} % Description(s)
        \item {More projects can be found at my github \href{https://github.com/Redcxx}{https://github.com/Redcxx}.}
      \end{cvitems}
    }
  
    


%---------------------------------------------------------
\end{cventries}
