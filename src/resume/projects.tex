%-------------------------------------------------------------------------------
%	SECTION TITLE
%-------------------------------------------------------------------------------
\cvsection{Projects}


%-------------------------------------------------------------------------------
%	CONTENT
%-------------------------------------------------------------------------------
\begin{cventries}

%---------------------------------------------------------
  \cventry
    {Sole Developer - HTML/CSS, Javascript, and GLSL} % Role
    {Personal Website} % Event
    {London, UK} % Location
    {Jun. 2020 - Present} % Date(s)
    {
      \begin{cvitems} % Description(s)
        \item {Written websites from scratch using various technologies such as \href{https://reactjs.org/}{reactJs}, \href{https://nextjs.org/}{nextJs}, \href{https://threejs.org/}{threeJs}, \href{https://sass-lang.com/}{sass},  \href{https://postcss.org/}{postcss}, and \href{https://pugjs.org/api/getting-started.html}{pugJs}.}
        \item {\href{https://weilueluo.com}{https://weilueluo.com} includes a \href{https://weilueluo.com}{3D landing page}, \href{https://weilueluo.com/about.html}{about page}, and \href{https://weilueluo.com/rss.html}{RSS feeds page}.}
        \item {Deployed an \href{https://developer.mozilla.org/en-US/docs/Web/HTTP/CORS}{CORS} proxy server for RSS feeds using AWS Lambda.}
        \item {Project can be found at \href{https://github.com/Redcxx/personal-website}{https://github.com/Redcxx/personal-website}.}
      \end{cvitems}
    }

    \cventry
      {Sole Developer - Python} % Job title
      {\href{https://www.noghost.co.uk/}{NoGhost} ML Workflow Enhancement} % Organization
      {London, UK} % Location
      {Jun. 2022 - Present - Sep. 2022} % Date(s)
      {
        \begin{cvitems} % Description(s) of tasks/responsibilities
          % \item {\textit{NoGhost is a animation studio based in London.}}
          \item {Assigned as the master project.}
          \item {Implement a number of state-of-the-art ML papers such as \href{https://arxiv.org/abs/1505.04597}{U-Net} \href{https://arxiv.org/pdf/1611.07004.pdf}{Pix2pix}, \href{https://arxiv.org/abs/1808.03240}{AlacGAN} and \href{https://esslab.jp/~ess/en/research/sketch/}{Sketch Simp} to aid artists' development process.}
          \item {Built the preprocessing, training and inference pipeline from scratch to manipulate raw data.}
        \end{cvitems}
      }

    
        %---------------------------------------------------------
  \cventry
    {Lead Developer - HTML/CSS, Javascript, and Python} % Job title
    {Collaborate Live} % Organization
    {Manchester, UK} % Location
    {Jan. 2019 - Mar. 2019} % Date(s)
    {
      \begin{cvitems} % Description(s)
        \item {Led a team of 7 students to build a web application for a code-based interview. Designed and implemented all core functionalities including real-time chatting, coding, and drawing with collaborators, supports online code compilation and web-embed terminal.}
        \item {Awarded 1\textsuperscript{st} place in most appealing idea and highest technical quality prize out of 37 teams.}
      \end{cvitems}
    }
    
%---------------------------------------------------------
  \cventry
    {Sole Developer - Python} % Role
    {Machine Learning with Pytorch} % Event
    {London, UK} % Location
    {Nov. 2020 - Present} % Date(s)
    {
      \begin{cvitems} % Description(s)
        \item {\textbf{Time Series Prediction} Generate Monophonic and Polyphonic piano music via various models \& techniques, including LSTM, skip connection \& attention. Report can be found at \href{https://github.com/Redcxx/Third-Year-Project-Report}{\textit{https://github.com/Redcxx/Third-Year-Project-Report}} and music samples can be found at \href{https://soundcloud.com/kallzvx/sets}{\textit{https://soundcloud.com/kallzvx/sets}}.}
        \item {\textbf{Image Classification} Transfer learning based on a pre-trained resnet, over 90\% accuracy with 10 classes of 1000 training samples.}
        \item {\textbf{Image Denosing} Built a Encoder-Decoder based network, result PSNR improved  by 100\%(15-30) compared to traditional Gaussian denosing technique.}
        \item {\textbf{Other} Familiar with linear/polynomial linear/logistic regression/classification; clustering algorithms like KMeans and Meanshift; SVM, kSVM, and wrote a simple network network from scratch.}
      \end{cvitems}
    }
    
  %---------------------------------------------------------
  % \cventry
  %   {Lead Developer - C\#} % Role
  %   {First Person Virtual Reality Game - Bouncing Ray} % Event
  %   {London, UK} % Location
  %   {Mar. 2022 - Apr. 2022} % Date(s)
  %   {
  %     \begin{cvitems} % Description(s)
  %       \item {Lead a team of 4 to build a first person Virtual Reality game based on the ubiq framework (\href{https://github.com/UCL-VR/ubiq}{\textit{https://github.com/UCL-VR/ubiq}}). Project can be found at \href{https://github.com/HengyiWang/COMP0113-BouncingRay}{\textit{https://github.com/HengyiWang/COMP0113-BouncingRay}}.}
  %     \end{cvitems}
  %   }


  
  %---------------------------------------------------------
  \cventry
    {Sole Developer - Python} % Role
    {Local Nameserver} % Event
    {London, UK} % Location
    {Feb. 2022 - Mar. 2022} % Date(s)
    {
      \begin{cvitems} % Description(s)
        \item {Written a local nameserver that able to resolve any IPv4 address according to \href{https://datatracker.ietf.org/doc/html/rfc1034}{RFC 1034} and \href{https://datatracker.ietf.org/doc/html/rfc1035}{RFC 1035} specification. The project can be found at \href{https://github.com/Redcxx/local-nameserver}{\textit{https://github.com/Redcxx/local-nameserver}}.}
      \end{cvitems}
    }
  
  %---------------------------------------------------------
  % \cventry
  %   {Sole Developer - Python} % Role
  %   {Poisson Image Editing} % Event
  %   {London, UK} % Location
  %   {Dec. 2021 - Jan. 2022} % Date(s)
  %   {
  %     \begin{cvitems} % Description(s)
  %       \item {Implemented the \href{https://www.cs.jhu.edu/~misha/Fall07/Papers/Perez03.pdf}{Poisson Image Editing paper}. Including techniques such as naive filling; import \& mixing gradient; texture flattening, colour isolation \& illumination. The project can be found at \href{https://github.com/Redcxx/poisson-image-editing}{\textit{https://github.com/Redcxx/poisson-image-editing}}.}
  %     \end{cvitems}
  %   }

  %---------------------------------------------------------
  \cventry
    {Sole Developer - Java} % Role
    {Propositional Logic Utils} % Event
    {Manchester, UK} % Location
    {Oct. 2019 - Nov. 2019} % Date(s)
    {
      \begin{cvitems} % Description(s)
        \item {A tool for parsing and manipulating propositional logic formulas, supports conversion to normal forms, push negations, tautology and contradiction checks and truth table generation. The project can be found at \href{https://github.com/Redcxx/PropositionalLogicUtils}{https://github.com/Redcxx/PropositionalLogicUtils}.}
      \end{cvitems}
    }

  %---------------------------------------------------------
  \cventry
    {Sole Developer - Python} % Role
    {Image Downloader} % Event
    {Manchester, UK} % Location
    {Jun. 2019 - Aug. 2019} % Date(s)
    {
      \begin{cvitems} % Description(s)
        \item {Designed and implemented an API supports automatic login, multithreading downloading, various format parsing and filter images, along with a graphical user interface. The project can be found at \href{https://github.com/Redcxx/Pikax}{\textit{https://github.com/Redcxx/Pikax}}.}
      \end{cvitems}
    }
  
    
  %---------------------------------------------------------
  \cventry
  {Sole Developer - Python} % Job title
  {Photographic Mosaic} % Organization
  {Manchester, UK} % Location
  {Aug. 2019 - Aug. 2019} % Date(s)
  {
    \begin{cvitems} % Description(s)
      \item {Built a library for generating photographic mosaic, capable of processing and matching over 30k images within 3 minutes, supports various RGB-based and LAB-based color difference algorithms. The project can be found at \href{https://github.com/Redcxx/Mosaic-Pics}{https://github.com/Redcxx/Mosaic-Pics}.}
    \end{cvitems}
  }

  %---------------------------------------------------------
  \cventry
  {Sole Developer - Java} % Job title
  {MP3 Music Player} % Organization
  {Manchester, UK} % Location
  {Mar. 2019 - Jun. 2019} % Date(s)
  {
    \begin{cvitems} % Description(s)
      \item {A simple MP3 player with GUI, supports functionalities, such as play, pause, restart, and upload custom songs. The project can be found at \href{https://github.com/Redcxx/MP3_Music_Player}{https://github.com/Redcxx/MP3\_Music\_Player}}
    \end{cvitems}
  }


%---------------------------------------------------------
\cventry
{} % Role
{Other} % Event
{} % Location
{} % Date(s)
{
  \begin{cvitems} % Description(s)
    \item {More projects can be found at my github \href{https://github.com/Redcxx}{\textit{https://github.com/Redcxx}}.}
  \end{cvitems}
}


\end{cventries}
