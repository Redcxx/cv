%-------------------------------------------------------------------------------
%	SECTION TITLE
%-------------------------------------------------------------------------------
\cvsection{项目}


%-------------------------------------------------------------------------------
%	CONTENT
%-------------------------------------------------------------------------------
\begin{cventries}

%---------------------------------------------------------
  \cventry
    {个人开发者 - HTML/CSS,Javascript,和 GLSL} % Role
    {个人网站} % Event
    {伦敦, 英国} % Location
    {六月 2020 - 现在} % Date(s)
    {
      \begin{cvitems} % Description(s)
        \item {从头搭建了两个个人网站,运用各类技术和第三方库如 \href{https://reactjs.org/}{reactJs},\href{https://nextjs.org/}{nextJs},\href{https://threejs.org/}{threeJs},\href{https://sass-lang.com/}{sass},\href{https://postcss.org/}{postcss} 和 \href{https://pugjs.org/api/getting-started.html}{pugJs}。}
        \item 前往 \url{https://luoweilue.com} (简单一点) 或者 \url{https://weilueluo.com} (好玩一点)查看。
      \end{cvitems}
    }
    
%---------------------------------------------------------
  \cventry
    {个人开发者 - Python} % Role
    {基于Pytorch的机器学习} % Event
    {伦敦, 英国} % Location
    {十一月 2020 - 现在} % Date(s)
    {
      \begin{cvitems} % Description(s)
      \item {\textbf{时间序列预测}\, 运用各类模型生成单/多声道的钢琴音乐包括LSTM,残差连接和注意力机制。可在 \href{https://github.com/Redcxx/Third-Year-Project-Report}{\textit{https://github.com/Redcxx/Third-Year-Project-Report}}查看报告,生成的音乐样本可在\href{https://soundcloud.com/kallzvx/sets}{\textit{https://soundcloud.com/kallzvx/sets}}查看。}
      \item {\textbf{图像分类}\, 基于Resnet50的迁移学习,用1000+图片分类10个标签,准确率达90\%。}
      \item {\textbf{图像降噪}\, 搭建Encoder-Decoder架构模型,相比于传统高斯降噪手段PSNR提高了100\%(15到30)。}
      \item {\textbf{其他}\, 熟悉线性/多项式 回归/分类算法;聚类算法如Kmeans和Meanshift;SVM,kSVM;从头手写过简单的神经网络。}
      \end{cvitems}
    }
    
  %---------------------------------------------------------
  \cventry
    {主要开发者 - C\#} % Role
    {第一人称虚拟现实射击游戏 - Bouncing Ray} % Event
    {伦敦, 英国} % Location
    {三月 2022 - 四月 2022} % Date(s)
    {
      \begin{cvitems} % Description(s)
        \item {带领4人团队用Unity基于Ubiq开发框架(\href{https://github.com/UCL-VR/ubiq}{\textit{https://github.com/UCL-VR/ubiq}})编写了一个第一人称虚拟现实射击游戏。项目可在 \href{https://github.com/HengyiWang/COMP0113-BouncingRay}{\textit{https://github.com/HengyiWang/COMP0113-BouncingRay}}查看。}
      \end{cvitems}
    }
  

  %---------------------------------------------------------
  \cventry
    {个人开发者 - Python} % Role
    {本地DNS服务器} % Event
    {伦敦, 英国} % Location
    {二月 2022 - 三月 2022} % Date(s)
    {
      \begin{cvitems} % Description(s)
        \item {根据\href{https://datatracker.ietf.org/doc/html/rfc1034}{RFC 1034} 和 \href{https://datatracker.ietf.org/doc/html/rfc1035}{RFC 1035}标准,编写了一个本地DNS服务器,可以用于解析任意IPv4地址。项目可在 \href{https://github.com/Redcxx/local-nameserver}{\textit{https://github.com/Redcxx/local-nameserver}}查看。}
      \end{cvitems}
    }
    
  %---------------------------------------------------------
  \cventry
    {个人开发者 - Python} % Role
    {泊松图像编辑} % Event
    {伦敦, 英国} % Location
    {十二月 2021 - 一月 2022} % Date(s)
    {
      \begin{cvitems} % Description(s)
        \item {实现 \href{https://www.cs.jhu.edu/~misha/Fall07/Papers/Perez03.pdf}{泊松图像编辑论文} 里的所有方法包括: 基础填充, 输入和混合梯度,纹理平滑,颜色孤立,和颜色光照。项目可在 \href{https://github.com/Redcxx/poisson-image-editing}{\textit{https://github.com/Redcxx/poisson-image-editing}}查看。}
      \end{cvitems}
    }


  %---------------------------------------------------------
  \cventry
    {个人开发者 - Java} % Role
    {命题逻辑工具} % Event
    {曼切斯特, 英国} % Location
    {十月 2019 - 十一月 2019} % Date(s)
    {
      \begin{cvitems} % Description(s)
        \item {一个用于解析和操作命题逻辑公式的工具,支持功能如范式转换、推否定、重言式和矛盾检查以及生成真值表。项目可在\href{https://github.com/Redcxx/PropositionalLogicUtils}{https://github.com/Redcxx/PropositionalLogicUtils}查看。}
      \end{cvitems}
    }

%---------------------------------------------------------
  \cventry
    {个人开发者 - Python} % Role
    {图片下载器} % Event
    {曼切斯特, 英国} % Location
    {六月 2019 - 八月 2019} % Date(s)
    {
      \begin{cvitems} % Description(s)
        \item {设计并实现了一个图片网站(\href{https://www.pixiv.net/}{Pixiv})下载接口,支持自动登录,多线程下载,多种格式解析和按各类标签过滤,外加一个图形界面。项目可在 \href{https://github.com/Redcxx/Pikax}{\textit{https://github.com/Redcxx/Pikax}}查看。}
      \end{cvitems}
    }

  %---------------------------------------------------------
  \cventry
    {个人开发者 - Python} % Job title
    {马赛克照片} % Organization
    {曼切斯特, 英国} % Location
    {八月 2019 - 八月 2019} % Date(s)
    {
      \begin{cvitems} % Description(s)
        \item {搭建了一个运用海量照片生成马赛克照片的工具。可在3分钟内处理并匹配30k+照片,支持各种基于RGB和LAB色彩空间的色差算法,项目可在\href{https://github.com/Redcxx/Mosaic-Pics}{https://github.com/Redcxx/Mosaic-Pics}查看。}
      \end{cvitems}
    }

  %---------------------------------------------------------
  \cventry
    {个人开发者 - Java} % Job title
    {MP3音乐播放器} % Organization
    {曼切斯特, 英国} % Location
    {三月 2019 - 六月 2019} % Date(s)
    {
      \begin{cvitems} % Description(s)
        \item {一个带有 GUI 的 简易 MP3 播放器,支持各种基本功能如播放、暂停、重启和上传自定义歌曲等。是过往3年唯一完成该项目的学生。项目可在\href{https://github.com/Redcxx/MP3_Music_Player}{https://github.com/Redcxx/MP3\_Music\_Player}查看。}
      \end{cvitems}
    }
    
  %---------------------------------------------------------
  \cventry
    {主要开发者 - HTML/CSS,Javascript,和 Python} % Job title
    {Collaborate Live} % Organization
    {曼切斯特, 英国} % Location
    {一月 2019 - 三月 2019} % Date(s)
    {
      \begin{cvitems} % Description(s)
        \item {带领7个学生搭建了一个代码面试网站。设计并实现了全部功能,包括实时聊天,实时共同编辑代码,实时共同画图等,并支持在线编译和在线命令行。}
        \item {在37个组里脱颖而出,获得最受欢迎奖和最高技术奖。}
      \end{cvitems}
    }
  

  \cventry
    {个人开发者} % Role
    {其他} % Event
    {} % Location
    {} % Date(s)
    {
      \begin{cvitems} % Description(s)
        \item {更多项目可在我的Github上查看: \href{https://github.com/Redcxx}{\textit{https://github.com/Redcxx}}。}
      \end{cvitems}
    }
  
    


%---------------------------------------------------------
\end{cventries}
